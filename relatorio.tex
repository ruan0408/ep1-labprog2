\documentclass[11pt]{article}
\usepackage[brazilian]{babel}
\usepackage[utf8]{inputenc} %Deixa eu colocar letras com ascentos
\usepackage[T1]{fontenc}
\usepackage{amsmath}
\usepackage{color}

\title{Relatório - EP Fase 1 \\ Laboratório de Programação 2}
\author{Victor Sanches Portella \and Lucas Dário \and Ruan Costa}


\begin{document}

\maketitle

\section{Integrantes}

\begin{itemize}

\item Victor Sanches Portella - Nº USP: 7991152

\item Lucas Dário - Nº USP: 7990940

\item Ruan Costa - Nº USP: 7990929

\end{itemize}


\section{Introdução}

Neste relatório daremos uma breve explicação sobre as estruturas e classes
criadas nesse EP. Daremos explicações que facilitem a entendimento do leitor quanto ao entendimento do código.

Portanto, primeiramente, faremos uma explanação sobre o funcionamento geral do programa, e após isso faremos uma breve explicação sobre alguns elementos existentes.

\section{O programa}

A ideia do programa, assim como explicado no enunciado, é receber um programa escrito em uma linguagem similar à linguagem \emph{Assembly}, transformar esse programa em um vetor de instruções e simular a execução desse programa.

Para resolver esse problema, usamos Programação Orientada a Objeto em \emph{Perl}. Assim, para tal, criamos 3 classes: \textbf{\color{red}Comando}, \textbf{\color{red}Programa} e \textbf{\color{red}Máquina}.

A classe \textbf{\color{red}Comando} representa o comando em si. A classe \textbf{\color{red}Programa} representa o Montador, e é responsável por transformar o arquivo de entrada em um vetor de objetos \textbf{\color{red}Comando}. A classe \textbf{\color{red}Máquina} representa a máquina virtual em si, representando a simulação de um determinado programa. A seguir, uma explicação um pouco mais específica sobre cada classe.

\subsection{Classe Comando}

Cada objeto dessa classe representa um comando, e é representado por dois atributos:

\begin{itemize}

\item \textbf{Opcode({\color{red}Obrigatório}):} String que define a função em si. Esse código é o mesmo do que o Mnemonico usado pela definição do comando no código fonte.

\item \textbf{Valor({\color{red}Opcional}):} Valor do argumento, caso exista. Se não existir, o valor desse atributo é \textbf{\color{blue}undef}. O valor, nos comandos implementados, pode ser interpretado como uma \emph{String} no caso do Jump, ou como um número nos outros casos.

\end{itemize}

Além dissso, usamos essa classe para verificar alguns erros de sintaxe. Ao tentarmos criar um comando, temos uma tabela para verificar se o comando existe e, no caso de existir, se precisa ser acompanhado de um argumento.(\textbf{\color{red}Importante})Caso o comando não exista, ou precise de um argumento que não foi passado, o programa irá retornar \textbf{\color{blue}undef}.

\subsection{Classe Programa}

Essa classe representa o montador, e possui 3 atributos:

\begin{itemize}

\item \textbf{Vetor:} Possivelmente o argumento mais importante dessa classe. Esse é o vetor que representa o vetor de instruções, no nosso caso representado por objetos do tipo \textbf{\color{red}Comando}, de todas as linhas já interpretadas pelo próprio objeto.

\item \textbf{Posi:} Indica a posição da próxima posição vazia no vetor. Esse atributo não seria de fato necessário, existiriam modos de fazer o programa sem ele, mas ele facilita o entendimento.




\item \textbf{Label:} É uma tabela de hash, onde as {\color{red}chaves} são o nome do label, e os {\color{red}valores associados} são a posição no vetor para o qual esse label ``aponta''.\footnote{Isso é usado para implementar os comandos de \emph{Jump}.}

\end{itemize}

Um objeto dessa classe possui o método \textbf{{\color{red}interpretaLinha}()}, que recebe uma string, que seria uma linha de nosso programa, trata a linha, criando uam referencia no hash \textbf{Label}, caso tenha um label, e criando um \textbf{Comando} e o colocando no vetor, caso o comando exista de fato.


\subsection{Classe Máquina}

A classe Máquina representa a máquina virtual que irá executar o nosso código. Nessa implementação, fizemos com que cada objeto do tipo \textbf{\color{red}Máquina} esteja atrelado a um objeto do tipo \textbf{\color{red}Programa}. Ou seja, para cada programa que você queira executar, você cria um objeto. Assim, todo objeto dessa classe possui três atributos:

\begin{itemize}

\item \textbf{Dados:} Vetor que representa a pilha de dados do programa em execução.

\item \textbf{Mem:} Vetor que representa a memória do programa.

\item \textbf{Programa:} Objeto do tipo \textbf{\color{red}Programa}, usado como fonte do vetor de instruções.



\end{itemize}

Não existe implementação da pilha de chamada de função pois não foi usado e/ou implementado chamada de funções na linguagem interpretada.

O principal método desse objeto é a função \textbf{\color{red}executa()}, que irá simular a execução do programa. Caso aconteça algum erro, a execução será interrompida no momento da exceção e uma mensagem de erro será exibida.


\section{Modo de Execução}

Modo de exeução do programa:

\begin{verbatim}

	meu_pc$>./main.pl arquivo_de_entrada

\end{verbatim}

Sendo que no \textbf{aquivo\_de\_entrada} se encontra o código fonte a ser executado. Caso você queira enviar os comandos através da 
\emph{STDIN}, só omitir o nome do arquivo.



\end{document}
